\documentclass[letter]{article}
\usepackage{amsmath}
\usepackage{amsfonts}
\usepackage{amssymb}
\usepackage{ifthen}
\usepackage{fancyhdr}
\usepackage{enumitem}

%%%
% Set up the margins to use a fairly large area of the page
%%%
\oddsidemargin=.2in
\evensidemargin=.2in
\textwidth=6in
\topmargin=0in
\textheight=9.0in
\parskip=.07in
\parindent=0in
\pagestyle{fancy}

%%%
% Set up the header
%%%
\newcommand{\setheader}[6]{
	\lhead{{\sc #1}\\{\sc #2} ({\small \it \today})}
	\rhead{
		{\bf #3} 
		\ifthenelse{\equal{#4}{}}{}{(#4)}\\
		{\bf #5} 
		\ifthenelse{\equal{#6}{}}{}{(#6)}%
	}
}

%%%
% Set up some shortcut commands
%%%
\newcommand{\R}{\mathbb{R}}
\newcommand{\N}{\mathbb{N}}
\newcommand{\Z}{\mathbb{Z}}
\renewcommand{\d}{\mathrm{d}}
\newcommand{\Proj}{\mathrm{proj}}
\newcommand{\Perp}{\mathrm{perp}}
\newcommand{\proj}{\mathrm{proj}}
\newcommand{\Span}{\mathrm{span}}
\newcommand{\Null}{\mathrm{null}}
\newcommand{\Rank}{\mathrm{rank}}
\newcommand{\mat}[1]{\begin{bmatrix}#1\end{bmatrix}}

%%%
% This is where the body of the document goes
%%%
\begin{document}
	\setheader{Math 281-1}{Homework 2}{Due: Thursday, October 6}{}{}{}

	\begin{enumerate}
		\item A rocket follows a straight path.  Its position along the path is $t^2$ 
			meters from the origin at time $t$.  The radius of the exhaust pipe at time
			$t$ is $r(t) = 8-t^{1/3}$.
			\begin{enumerate}
				\item The EPA wants an estimate of the total volume of exhaust from $t=0$ to $t=8$
					seconds.  They request you estimate this volume by sampling the radius of the 
					exhaust pipe and the length of the exhaust column at
					$8$ regularly spaced {\bf times}.

					Write and evaluate a summation for the EPA's requested estimate.

				\item Use an integral to produce a more accurate estimate of the total amount of exhaust (you 
					may assume the volume of exhaust in an exhaust column of height $h$ and radius $r$ is 
					$\pi r^2h$).
			\end{enumerate}
		\item For this problem, you will be using {\sc Matlab}/{\sc Octave}.
			Please include a printout of your code along with your assignment.
			\begin{enumerate}
				\item Let $f(x)=x^2$.  Set up a right-endpoint Riemann sum to
					estimate the value of $\displaystyle \int_0^{3} f(x)\,\d x$.
				\item Use {\sc Matlab}/{\sc Octave} to evaluate the
					Riemann sum from part (a) using $10$, $100$, and $1000$ intervals.
					Report the deviation from the exact integral for each case.
				\item Consider the \emph{parameterization} $p:\R\to\R$ of the real numbers
					given by $p(t)=t^{1/3}$.  Find an interval $[a,b]$ such that
					$p([a,b])=[0,3]$.  (If you're unfamiliar with
					this notation, for a set $X$, $p(X)$ is the \emph{image}
					of $X$ under $p$.  That is, $p(X)= \{y:y=p(x)\text{ for some }x\in X\}$).
				\item The command {\tt stairs(xs, ys)} can be used to graph a
					step function in {\sc Matlab}/{\sc Octave}.  Perfect
					for visualizing Riemann sums!  We're going to compare
					two ways of approximating $f$.  Use {\tt stairs} to
					plot $(t,f(t))$ for $50$ equally spaced points in $[0,3]$.
					Again, use {\tt stairs} to plot $(p(t), f\circ p(t))$ for
					$50$ equally spaced points in your interval $[a,b]$ from above.

					How do the domain and range of your $t$-values compare?
					How do the $x$ and $y$ values for each of your plots compare?
				\item Set up a sum that estimates the area under the curve $(p(t), f\circ p(t))$
					when $t$ ranges over your previously-established interval $[a,b]$.  Use
					{\sc Matlab}/{\sc Octave} to compute the value of this sum for
					$10$, $100$, and $1000$ regularly spaced $t$ values.  Should you
					expect these estimates to be close to $\displaystyle \int_0^3 f(x)\,\d x$?
					How do your estimates compare to part (b)?
				\item Set up and evaluate an integral to find the exact area under the curve $(p(t),f\circ p(t))$
					where $t$ ranges over your previously-established interval $[a,b]$.
					The bounds of your integral should be $a$ and $b$.
					(\emph{Hint: set up a general sum to estimate the area and then take
					a limit as $\Delta t\to 0$}).
			\end{enumerate}
		\item You travel, starting from the origin and heading in the positive direction, along the
			$x$-axis with a speed given by $s(x)=\sqrt{x}$ units per second, where $x$ is your position along
			the $x$-axis.  You sample the height of a function as you travel and discover
			$h(t)=(2-t)^2$ where $t$ is time in seconds.
			\begin{enumerate}
				\item How long does it take you to get to $x=10$?
				\item Give a relationship between $x$ and $t$.
				\item Reparameterize $h$ in terms of $x$.
				\item Write an integral formula for the area under $h$ from $x=0$ to $x=10$.
			\end{enumerate}
	\end{enumerate}

	
	Recall that $\text{Work}=\vec F\cdot \vec d$ where $\vec F$ is a force vector and
	$\vec d$ is a displacement vector.
	
	\begin{enumerate}[resume]
		\item A turbulent river pushes a particle at the point $(x,y)$ with a force
			\[
				\vec F(x,y) = (-yx,x).
			\]
			You are pushing a box through the river from $(0,0)$ to $(1,1)$ along a 
			straight path.
			\begin{enumerate}
				\item Does $\vec F(0,0)\cdot (1,1)$ give you the total work done?
					How about $\vec F(1,1)\cdot (1,1)$?  Why or why not?
				\item Suppose you take tiny steps of size $\sqrt{2}/n$ on your way
					from $(0,0)$ to $(1,1)$.  Write a summation that
					represents the total amount of work done traveling
					from $(0,0)$ to $(1,1)$.  If it's not the case already,
					rewrite your summation so it involves a dot product.
				\item Write down the integral that results from your summation in
					(b) when you let $n\to \infty$.
				\item What is the total work done?
			\end{enumerate}
		
		\item A vector moves along the path $\vec r(t) = \mat{\sin t\\ 2\cos t\\ t}$ through a force field given by
			$\vec F(x,y,z)=\mat{yz\\ xz\\ xy}$.  Find the total work done between $t=0$ and $t=\pi$.

	\end{enumerate}

\end{document}
