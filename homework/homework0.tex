\documentclass[letter]{article}
\usepackage{amsmath}
\usepackage{amsfonts}
\usepackage{amssymb}
\usepackage{ifthen}
\usepackage{fancyhdr}

%%%
% Set up the margins to use a fairly large area of the page
%%%
\oddsidemargin=.2in
\evensidemargin=.2in
\textwidth=6in
\topmargin=0in
\textheight=9.0in
\parskip=.07in
\parindent=0in
\pagestyle{fancy}

%%%
% Set up the header
%%%
\newcommand{\setheader}[6]{
	\lhead{{\sc #1}\\{\sc #2} ({\small \it \today})}
	\rhead{
		{\bf #3} 
		\ifthenelse{\equal{#4}{}}{}{(#4)}\\
		{\bf #5} 
		\ifthenelse{\equal{#6}{}}{}{(#6)}%
	}
}

%%%
% Set up some shortcut commands
%%%
\newcommand{\R}{\mathbb{R}}
\newcommand{\N}{\mathbb{N}}
\newcommand{\Z}{\mathbb{Z}}
\newcommand{\Proj}{\mathrm{proj}}
\newcommand{\Perp}{\mathrm{perp}}
\newcommand{\proj}{\mathrm{proj}}
\newcommand{\Span}{\mathrm{span}}
\newcommand{\Null}{\mathrm{null}}
\newcommand{\Rank}{\mathrm{rank}}
\newcommand{\mat}[1]{\begin{bmatrix}#1\end{bmatrix}}
\newcommand{\makescale}{%
	\par
	\begin{center}
	\begin{minipage}{.95\textwidth}
	{\par
	\begin{minipage}{1.5in}\footnotesize No knowledge\end{minipage}\hfill
	\begin{minipage}{1.5in}\footnotesize \centerline{I know the basics}\end{minipage}\hfill
	\begin{minipage}{1.5in}\footnotesize \raggedleft Advanced Knowledge\end{minipage}
	\par
	\vspace{.2cm}
	\hspace{2em}1\dotfill 2\dotfill3\dotfill4\dotfill5\hspace{2em}
	\par
	}
	\vspace{.6cm}
	{\par
	\begin{minipage}{1.7in}\raggedright \footnotesize I'd like a lot of practice and class time focused on this concept\end{minipage}\hfill
	\begin{minipage}{1.5in}\footnotesize \begin{center}Spending some class time on this would be good\end{center}\end{minipage}\hfill
	\hspace{.2in}\begin{minipage}{1.5in}\footnotesize \raggedleft With a list of topics, I can study on my own\end{minipage}
	\par
	\vspace{.2cm}
	\hspace{2em}1\dotfill 2\dotfill3\dotfill4\dotfill5\hspace{2em}
	\par
	}
	\end{minipage}
	\end{center}
}

%%%
% This is where the body of the document goes
%%%
\begin{document}
	\setheader{Math 281-1}{Homework 0}{Due: Tuesday, September 20}{}{}{}
	In this class, we'll be learning a lot of math, but we'll also be \emph{thinking about thinking} 
	and \emph{learning about learning}.  
	I'd like to understand your thoughts in this regard.  As such, please type your responses to the following questions. 
	Your combined answers should be no longer than 2 pages.
	\begin{enumerate}
		\item What type of thinking do you expect to do in a math
		class? How is this type of thinking similar or different
		to the types of thinking you might do in classes for
		other subjects?  Please give some examples.

		\item What roles do \emph{you}, \emph{your instructor}, and 
		\emph{your peers}
		each play in the process of learning?

		\item In math, how can you tell if you're right or wrong?

		\item In math, is an answer always either right or wrong?  Please explain.

		\item Professors in the English department list
		``evaluate multiple perspectives,'' ``reframe
		questions and issues,'' and ``examine central issues
		and assumptions'' as goals for their courses.	Are these
		goals relevant to math? Please explain.

		\item Other thoughts?
	\end{enumerate}

	\newpage

	For each topic, please indicate your current level of experience and how much 
	support you'd like when learning or reviewing the material.

	\vspace{.4cm}
	{\bf Complex Numbers}
	\makescale
	
	\vspace{.8cm}
	{\bf Vectors}
	\makescale
	
	\vspace{.8cm}
	{\bf Computer Programming}
	\makescale
	
	\vspace{.8cm}
	{\bf Proofs}
	\makescale
\end{document}
