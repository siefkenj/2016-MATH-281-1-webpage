\documentclass[letter]{article}
\usepackage{amsmath}
\usepackage{amsfonts}
\usepackage{amssymb}
\usepackage{ifthen}
\usepackage{fancyhdr}
\usepackage{enumitem}

%%%
% Set up the margins to use a fairly large area of the page
%%%
\oddsidemargin=.2in
\evensidemargin=.2in
\textwidth=6in
\topmargin=0in
\textheight=9.0in
\parskip=.07in
\parindent=0in
\pagestyle{fancy}

%%%
% Set up the header
%%%
\newcommand{\setheader}[6]{
	\lhead{{\sc #1}\\{\sc #2} ({\small \it \today})}
	\rhead{
		{\bf #3} 
		\ifthenelse{\equal{#4}{}}{}{(#4)}\\
		{\bf #5} 
		\ifthenelse{\equal{#6}{}}{}{(#6)}%
	}
}

%%%
% Set up some shortcut commands
%%%
\newcommand{\R}{\mathbb{R}}
\newcommand{\N}{\mathbb{N}}
\newcommand{\Z}{\mathbb{Z}}
\newcommand{\Proj}{\mathrm{proj}}
\newcommand{\Perp}{\mathrm{perp}}
\newcommand{\proj}{\mathrm{proj}}
\newcommand{\Span}{\mathrm{span}}
\newcommand{\Null}{\mathrm{null}}
\newcommand{\Rank}{\mathrm{rank}}
\newcommand{\mat}[1]{\begin{bmatrix}#1\end{bmatrix}}

%%%
% This is where the body of the document goes
%%%
\begin{document}
	\setheader{Math 281-1}{Homework 4}{Due: Thursday, October 20}{}{}{}
	\begin{enumerate}

		\item Given a curve $\mathcal S\subset \R^n$, the \emph{curvature} of $\mathcal S$ at the point $\vec p\in \mathcal S$ is the
			magnitude of the acceleration when passing through $\vec p$ at unit speed (following the curve $S$).
			That is, if $\vec r(t)$ is an arc-length parameterization of $S$ and $\vec r(t_0) = \vec p$, 
			then the curvature of $\mathcal S$ at the point $\vec p$ would be $\|\vec r\,''(t_0)\|$.

			\begin{enumerate}
				\item Let $\mathcal{S}_r\subset \R^2$ be a circle of radius $r$ centered at the origin.  
					Compute the curvature 
					of $\mathcal{S}_r$ at the point $\vec p=(r,0)$.
				\item The following points lie on the curve $\mathcal C\subset \R^2$.

					\begin{center}
					\begin{tabular}{c|c}
						$x$&$y$\\
						\hline
						0.7&0.49\\
						0.8&0.64\\
						0.9&0.81\\
						1&1\\
						1.1&1.21\\
						1.2&1.44\\
						1.3&1.69
					\end{tabular}
				\end{center}
				\begin{enumerate}
					\item Suppose that $\vec r$ is an arc-length parameterization
						of $\mathcal C$ and that $\vec r(t_0)=(1,1)$.  Estimate $\vec r\,'(t_0)$.
						(\emph{Hint: make sure your vector has the correct length!})
					\item Estimate the curvature of $\mathcal C$ at $(1,1)$.
				\end{enumerate}
			\end{enumerate}
		\item For this problem, we will be using {\sc Matlab}/{\sc Octave}, though not every part
			requires programming.  Let $\mathcal P$ be the part
			of the parabola $1-x^2$ that is above the $x$-axis and consider the following parameterizations 
			of $\mathcal P$:
			\[
				\vec r(t) = \mat{t\\1-t^2}\qquad \text{and}\qquad \vec p(t)=\mat{\sin(\pi t/2)\\1-\sin^2(\pi t/2)}.
			\]
			\begin{enumerate}
				\item Find the domains of $\vec r$ and $\vec p$ such that they are indeed parameterizations of $\mathcal P$.
				\item Plot a numerical estimate of the speed of $\vec r$ and $\vec p$ vs\mbox{.} time.  You may find
					the following {\sc Matlab}/{\sc Octave} tips helpful:

					If you have a list {\tt x=[1, 4, 9, 16, 25]}, for example,
					and you would like to get a list of the consecutive
					differences between entries in {\tt x}, you can use the command
					\begin{center}
						{\tt x(:, 2:length(x)) - x(:, 1:(length(x)-1))}
					\end{center}
					If you have a list {\tt vecs} whose \emph{columns} are vectors and you'd like to get a list
					containing the lengths of those vectors, you can use the command
					\begin{center}
						{\tt sqrt(sum(vecs .* vecs, 1))}
					\end{center}
					The extra {\tt 1} in the {\tt sum} command tells {\sc Matlab}/{\sc Octave} to 
					sum along the columns (the command {\tt sum(x, 2)} would sum along the rows).

					\emph{Hint: it will be worth your time to define {\sc Matlab}/{\sc Octave}
						functions for $\vec r$ and $\vec p$ 
					 for use later in the problem.  Also, make sure you
					understand why the example code above works before you use it.}
				\item Plot the arc length of the path traversed by $\vec r$ and $\vec p$ with respect
					to time.  
						For which $t_0$ do you expect $\mathrm{arclen}_0^{t_0}(\vec r) = \mathrm{arclen}_0^{t_0}(\vec p)$?
					Explain.  (\emph{Hint: don't
						look for a formula for arc length!  Use {\sc Matlab}/{\sc Octave} to create a list
					whose $i$th item is the arc length up to the $i$th time step.})
				\item Inverting functions given by formulas is hard, but inverting functions as a concept is 
					easy---you just switch the $x$ and $y$ coordinates!  In {\sc Matlab}/{\sc Octave}
					we have easy access to $x$ and $y$ coordinates, however we don't have access to 
					\emph{all} $x$ and $y$ coordinates.  The solution is to estimate the points we
					don't have based on those we do.  This process is called \emph{interpolation}.

					We will create an approximation of $f(x)=x^2$ and its inverse on the interval
					$[0,10]$ using 11 regularly spaced points.  Create two lists, {\tt xs = 0:1:10}
					and {\tt ys = xs .* xs}. We will define approximations to $f$ and $f^{-1}$ using the 
					{\tt interp1} command.  Create two new functions with the following code:

					{ \tt fapprox = @(x) (interp1(xs, ys, x, 'spline'))
					}
					
					{ \tt fiapprox = @(x) (interp1(ys, xs, x, 'spline'))
					}

					The {\tt 'spline'} argument tells {\sc Matlab}/{\sc Octave} to do a smooth
					approximation using polynomials
					rather than jaggedy approximation with lines or step functions.

					\begin{enumerate}
						\item Using at least 1001 equally spaced points in the interval $[0,10]$,
							graph $f$ and {\tt fapprox} on the same plot.  On a separate plot,
							graph $f^{-1}$ and {\tt fiapprox}.  Where do the functions
							exactly match their approximations?  Why?  (\emph{Hint: {\tt plot(x1s, y1s, x2s, y2s)}
							can be used to plot two functions in the same graph}.)
						\item Using at least 1001 equally spaced points in the interval $[0,10]$,
							plot {\tt fiapprox $\circ $ fapprox} and {\tt fapprox $\circ$ fiapprox}.
							If {\tt fiapprox} and {\tt fapprox} were perfect inverses of each other,
							what graph should you get?  Why doesn't your graph look like that?
						\item Define two new functions {\tt fgoodapprox} and {\tt figoodapprox}
							using 51 equally spaced points between $[0,10]$ as the basis for your
							approximations.  Then, graph {\tt figoodapprox $\circ $ fgoodapprox}
							and {\tt fgoodapprox $\circ $ figoodapprox}.  Is this closer to what
							you expected?
					\end{enumerate}
				\item Let $\vec a$ be the arc-length parameterization of
					$\mathcal P$ with $\vec a(0) = (-1,0)$.
					Define $\vec a$ as a function in {\sc Matlab}/{\sc Octave}.
					On separate plots, plot $\|\vec r\, ''(t)\|$, $\|\vec p\, ''(t)\|$, and
					$\|\vec a\, ''(t)\|$.  You previously found values $t_0$ where $\vec r(t_0)=\vec p(t_0)$.
					At these values, should $\|\vec r\,''(t_0)\| = \|\vec p\,''(t_0)\|$?  Explain.

				\item Let $c:\mathcal P\to \R$ be the function that takes a point on $\mathcal P$ and
					returns the curvature at that point.  Use {\sc Matlab}/{\sc Octave} to
					plot $c\circ \vec p$, $c\circ \vec r$, and $c\circ \vec a$. 
					You previously found values $t_0$ where $\vec r(t_0)=\vec p(t_0)$.
					At these values, should $c\circ r(t_0) = c\circ \vec p(t_0)$?  Explain.





			\end{enumerate}
	\end{enumerate}

\end{document}
