\documentclass[letter]{article}
\usepackage{amsmath}
\usepackage{amsfonts}
\usepackage{amssymb}
\usepackage{ifthen}
\usepackage{fancyhdr}
\usepackage{matlab-prettifier}

\usepackage[no-math]{fontspec}
\setmonofont{Fira Mono}
\setmainfont[BoldFont={Fira Sans Medium}]{Fira Sans Light}

%%%
% Set up the margins to use a fairly large area of the page
%%%
\oddsidemargin=.2in
\evensidemargin=.2in
\textwidth=6in
\topmargin=0in
\textheight=9.0in
\parskip=.07in
\parindent=0in
\pagestyle{fancy}

%%%
% Set up the header
%%%
\newcommand{\setheader}[6]{
	\lhead{{\sc #1}\\{\sc #2} ({\small \it \today})}
	\rhead{
		{\bf #3} 
		\ifthenelse{\equal{#4}{}}{}{(#4)}\\
		{\bf #5} 
		\ifthenelse{\equal{#6}{}}{}{(#6)}%
	}
}

%%%
% Set up some shortcut commands
%%%
\newcommand{\R}{\mathbb{R}}
\newcommand{\N}{\mathbb{N}}
\newcommand{\Z}{\mathbb{Z}}
\newcommand{\Proj}{\mathrm{proj}}
\newcommand{\Perp}{\mathrm{perp}}
\newcommand{\proj}{\mathrm{proj}}
\newcommand{\Span}{\mathrm{span}}
\newcommand{\Null}{\mathrm{null}}
\newcommand{\Rank}{\mathrm{rank}}
\newcommand{\mat}[1]{\begin{bmatrix}#1\end{bmatrix}}

\newenvironment{code}{\begin{lstlisting}[style=Matlab-editor,escapechar=`]}{\end{lstlisting}}


%%%
% This is where the body of the document goes
%%%
\begin{document}
\lhead{Guide to {\sc Matlab}/{\sc Octave} for Multivariable Calculus}

	{\sc Matlab} and {\sc Octave} are both programming environments
	for doing numerical mathematics and working with vectors.  {\sc Matlab}
	is used in many scientific and engineering environments, and 
	{\sc Octave} is an open-source version of {\sc Matlab}.  You may use
	either tools for this course.  To save some typing, I will use 
	{\sc Octave} to refer to {\sc Matlab} or {\sc Octave} from here on out.

	\section{The Basics}
	{\sc Octave} is designed to do \emph{interactive} mathematics.  That is,
	you can type mathematics and immediately get results.  Simple math
	operations work how you expect:
\begin{lstlisting}[style=Matlab-Pyglike,escapechar=`]
octave:1> 3*5
ans =  15
octave:2> 2*pi
ans =  6.2832
octave:3> 5^9
ans =  1953125
octave:4> sin(pi/3)
ans =  0.86603
\end{lstlisting}

	Keep in mind, everything you do in {\sc Octave} will be \emph{numerical}
	as opposed to \emph{exact}.  This makes {\sc Octave} great for experimenting,
	but after you've found an answer, you might want to algebraically reason
	it through.

	{\sc Octave} can also input and manipulate vectors.  To do so, use square brackets
	{\tt [ ]} and list the components of your vector.
\begin{lstlisting}[style=Matlab-Pyglike,escapechar=`]
octave:5> [1; 2; 3]
ans =

   1
   2
   3

octave:6> dot([1;2;3], [4;5;6])
ans =  32
	\end{lstlisting}
	If you use `{\tt ;}' between the components of your vector, it will be a column vector.
	If you use `{\tt ,}' between the components of your vector, it will be a row vector.
	In this class, we will primarily use column vectors.

	You can store values/vectors in variables with {\tt =}.
	\begin{lstlisting}[style=Matlab-Pyglike,escapechar=`]
octave:7> x = [1; 2]
x =

   1
   2

octave:8> y = [-2; 1]
y =

  -2
   1

octave:9> dot(x, y)
ans = 0
	\end{lstlisting}
	By default, {\sc Octave} will print out the contents of a variable after it is assigned.  To suppress 
	the printing, add a `{\tt ;}' to the end of the line.
	\begin{lstlisting}[style=Matlab-Pyglike,escapechar=`]
octave:10> x = [1; 2];
octave:11> y = [-2; 1];
octave:12> dot(x, y)
ans = 0
	\end{lstlisting}

	You can access and set components of vectors by doing \emph{vector}{\tt (}\emph{component}{\tt )}.
	\begin{lstlisting}[style=Matlab-Pyglike,escapechar=`]
octave:13> x=[5; -2; 7];
octave:14> x(1)
ans =  5
octave:15> x(2)
ans = -2
octave:16> x(2) = 14;
octave:17> x
x =

    5
   14
    7
	\end{lstlisting}

	\section{Functions}
	{\sc Octave} comes with many built-in functions, including {\tt sin}, {\tt cos}, {\tt tan},
	and their inverses {\tt asin}, {\tt acos}, {\tt atan}.  It also comes with many vector-based
	functions, like {\tt dot}.  However, you will often want to define your own functions.  If your
	function can be expressed in one line, the easiest way to define it is with the {\tt @()} syntax.
	For example, if we wanted a function $f:\R^3\to \R$ that summed the components of a vector,
	we might write:
\begin{lstlisting}[style=Matlab-Pyglike,escapechar=`]
octave:18> f = @(x) (x(1) + x(2) + x(3))
f =

@(x) (x (1) + x (2) + x (3))

octave:19> a=[1; 2; 3];
octave:20> f(a)
ans =  6
\end{lstlisting}
	When we defined $f$, the $x$ was a \emph{dummy variable}.  That is, it has no meaning outside the definition
	of the function and is only used as a name.  Therefore, it is equivalent to say:
	\begin{lstlisting}[style=Matlab-Pyglike,escapechar=`]
octave:18> f = @(t) (t(1) + t(2) + t(3))
f =

@(t) (t (1) + t (2) + t (3))

octave:19> a=[1; 2; 3];
octave:20> f(a)
ans =  6
	\end{lstlisting}

	\subsection{Math Operations}
	You've probably noticed already that the operations {\tt +}, {\tt -}, {\tt *}, {\tt \textasciicircum}, and {\tt /} are
	usable in {\sc Octave}.  If you use them with single numbers, these operations do
	what you expect.  However, they are actually \emph{vector} operations, not \emph{scalar} operations.
	The corresponding scalar operations are {\tt .+}, {\tt .-}, {\tt .*}, {\tt .\textasciicircum}, and {\tt ./} (they all
	have a `{\tt .}' in front).  When we learn matrix/vector operations, the distinction will
	make sense, but for now, just use a `{\tt .}' in front of these operations (you can omit the `{\tt .}'
	in front of {\tt +} and {\tt -} because vector addition and scalar addition are treated nearly identically
	in {\sc Octave}).

	Putting this into practice, if we wanted to define $f(x)=x^2$, we should write
	\begin{lstlisting}[style=Matlab-Pyglike,escapechar=`]
octave:21> f = @(x) (x.^2)
f =

@(x) (x .^ 2)

octave:22> f(5)
ans =  25
	\end{lstlisting}
	and not \verb| f = @(x) (x^2)|.

	\section{Graphing}
	There are two types of things you might want to graph: sets of vectors/points and functions.
	Let's start with points.

	The {\tt plot} command can be used for graphing, though the syntax might not quite be
	what you are used to.  {\tt plot} takes a list of $x$'s, $y$'s and a \emph{style}.  For example,
	\begin{lstlisting}[style=Matlab-Pyglike,escapechar=`]
octave:25> plot([0 1 2 3 4], [0 1 4 9 16], 'o')
	\end{lstlisting}
	will plot the points $(0,0)$, $(1,1)$, $(2,4)$, $(3,9)$, and $(4,16)$ putting a little circle
	at each point.  Replacing {\tt 'o'} with {\tt '.'} would put a dot at each point and
	using {\tt '-'} would connect each pair of consecutive points with a line segment.

	Since {\sc Octave} plots $(x,y)$ pairs in the usual way, if you've defined a function, you
	can plot it by evaluating the function on a set of $x$ values.
	\begin{lstlisting}[style=Matlab-Pyglike,escapechar=`]
octave:29> f = @(x) (x.^(1/2));
octave:30> xs = [0 1 2 3 4 5];
octave:31> plot(xs, f(xs), '.')
	\end{lstlisting}

	It can be tedious to type out $x$ values, so there is a shortcut syntax: \emph{start}{\tt :}\emph{step}{\tt :}\emph{stop}.
	So, to plot $f(x)=\sqrt{x}$ with points space $0.1$ apart on the interval $[0,5]$, you could run the command:
	\begin{lstlisting}[style=Matlab-Pyglike,escapechar=`]
octave:29> f = @(x) (x.^(1/2));
octave:30> xs = 0:.1:5;
octave:31> plot(xs, f(xs), '.')
	\end{lstlisting}

	\subsection{Graphing Vectors}
	What if we have a bunch of vectors that we'd like to plot?  Let's say we define a
	function $f:\R\to\R^2$ by
	\[
		f(t)=\mat{\sin t\\\cos t}.
	\]
	In order to plot such a function, we would need to get a hold of the $x$ and $y$ coordinates
	of the output.  
	Let's first examine the initial setup.
	\begin{lstlisting}[style=Matlab-Pyglike,escapechar=`]
octave:36> f = @(t) ([sin(t); cos(t)]);
octave:37> ts = 0:.25:pi;
octave:38> f(ts)
ans =

   0.00000   0.24740   0.47943   0.68164   0.84147   0.94898   0.99749   0.98399   0.90930   0.77807   0.59847   0.38166   0.14112
   1.00000   0.96891   0.87758   0.73169   0.54030   0.31532   0.07074  -0.17825  -0.41615  -0.62817  -0.80114  -0.92430  -0.98999
	\end{lstlisting}
	Here, the output of {\tt f(ts)} is actually a list of vectors.  We'd like to get a hold of the of the $x$ and $y$ coordinates
	of vectors in the list. This can be done with the \emph{vectors}{\tt (1,:)} and \emph{vectors}{\tt (2,:)} syntax.
	\begin{lstlisting}[style=Matlab-Pyglike,escapechar=`]
octave:37> f = @(t) ([sin(t); cos(t)]);
octave:38> ts = 0:.25:pi;
octave:39> vecs = f(ts);
octave:40> vecs(1,:)
ans =

   0.00000   0.24740   0.47943   0.68164   0.84147   0.94898   0.99749   0.98399   0.90930   0.77807   0.59847   0.38166   0.14112

octave:41> vecs(2,:)
ans =

   1.000000   0.968912   0.877583   0.731689   0.540302   0.315322   0.070737  -0.178246  -0.416147  -0.628174  -0.801144  -0.924302  -0.989992
	\end{lstlisting}
	We can see that \emph{vectors}{\tt (1,:)} picked off the $x$ coordinates and \emph{vectors}{\tt (2,:)}
	picked off the $y$ coordinates.  (What really happened was that we indexed a matrix using \emph{{\sc Octave}
	slice notation}.  That is, the {\tt 1} in  {\tt(1, :)} says take the first row and the {\tt :} says take all
	numbers in that row.  Can you guess how to grab only the first three numbers of the first row?)

	Now we can plot without issue:
	\begin{lstlisting}[style=Matlab-Pyglike,escapechar=`]
octave:37> f = @(t) ([sin(t); cos(t)]);
octave:38> ts = 0:.25:pi;
octave:39> vecs = f(ts);
octave:40> plot(vecs(1,:), vecs(2,:), '.')
	\end{lstlisting}

	\subsection{Multiple Graphs}
	There may be times when you'd like multiple graphs to show up on
	the same plot.  For that, you can use the {\tt hold on} and 
	{\tt hold off} commands.  The command {\tt hold on} prevents {\sc Octave}
	from creating a new plot, and therefore everything you plot will get pushed
	to the existing plot.  The command {\tt hold off} stops this behaviour.
	For example, to plot both $f(t) = (\sin t,\cos t)$ and $g(t)=(t,t^2)$ we
	could do:
	\begin{lstlisting}[style=Matlab-Pyglike,escapechar=`]
octave:37> f = @(t) ([sin(t); cos(t)]);
octave:38> g = @(t) ([t; t.^2]);
octave:39> ts = 0:.25:pi;
octave:40> vec_f = f(ts);
octave:41> vec_g = f(ts);
octave:42> plot(vec_f(1,:), vec_f(2,:), '-')
octave:43> hold on
octave:44> plot(vec_g(1,:), vec_g(2,:), '-')
octave:45> hold off
	\end{lstlisting}

	\section{Flow Control and Loops}
	Flow control is programming-speak for piecewise functions.  That is, functions that 
	have ``if'' statements in them.  Loops are ways to repeat a procedure without
	copying and pasting.


	We'll start with loops.  A loop in {\sc Octave} is written with the 
	\begin{lstlisting}[style=Matlab-Pyglike,escapechar=`]
for i=1:n,
    `\mlplaceholder{loop contents}`
end
	\end{lstlisting}
	syntax.

	For example, if we want the variable $x$ to be an accumulated sum of squares from $1^2$ to $15^2$ 
	(that is $x=1^2+2^2+\cdots+15^2$), we could write
	\begin{lstlisting}[style=Matlab-Pyglike,escapechar=`]
x=0;
for i=1:15,
    x = x + i.^2;
end
	\end{lstlisting}
	After executing this code $x$ will be the value $1240$.  The code works as follows: first we assign the value
	$0$ to $x$.  Then we enter the loop.  In the first iteration, when $i$ is 1, the new value of $x$ will be the current value,
	0, plus $1^2$.  The next time through the loop, $i$ is 2 and so we assign $x$ to be the current value, $1^2$, plus
	$2^2$, etc.. Once we have looped through with the value of $i$ being 15, we stop.

	If statements are written with the 
\begin{lstlisting}[style=Matlab-Pyglike,escapechar=`]
if `\mlplaceholder{variable}` == `\mlplaceholder{value}`,
    `\mlplaceholder{if contents}`
end
\end{lstlisting}[style=Matlab-Pyglike,escapechar=`]
	syntax (note the double equals). 

	For example, if we wanted $x$ to be the sum of the squares of only the even numbers between 1 and 15, we might write:
\begin{lstlisting}[style=Matlab-Pyglike,escapechar=`]
x = 0;
for i = 1:15,
    if round(i/2) == i/2,
        x = x + i.^2;
    end
end
\end{lstlisting}

	Here, {\tt round(i/2)==i/2} is true when {\tt i/2} has no decimal places and so it is even.  Thus, when that happens
	(and only when that happens) we add the value $i^2$ to $x$.

	\subsection{Accumulating Vectors}
	Often times, you'll want to use a loop to create a list of vectors.  For example,
	maybe we want to create a list of vectors on the unit circle, one for each degree.
	One way to do this is to do a loop and \emph{append} each vector created to a list.

	The syntax for appending looks a lot like nesting matrices.  Observe the following:
	\begin{lstlisting}[style=Matlab-Pyglike,escapechar=`]
octave:51> x=[1; 2];
octave:52> y=[2; 7];
octave:53> z=[-1; -1];
octave:54> [x y]
ans =

   1   2
   2   7

octave:55> [x y z]
ans =

   1   2  -1
   2   7  -1
	\end{lstlisting}
	We can use this syntax to accumulate a list of vectors.
	\begin{lstlisting}[style=Matlab-Pyglike,escapechar=`]
octave:51> x=[1; 2];
octave:52> y=[2; 7];
octave:53> z=[-1; -1];
octave:59> r = [x]
r =

   1
   2

octave:60> r = [r y]
r =

   1   2
   2   7

octave:61> r = [r z]
r =

   1   2  -1
   2   7  -1
	\end{lstlisting}
	Notice that one item gets added to {\tt r} each time.  This can be combined with a {\tt for}
	loop to generate the list of vectors we want.
	\begin{lstlisting}[style=Matlab-Pyglike,escapechar=`]
r = [];
for i=0:360,
    v = [cos(2*pi*i/360); sin(2*pi*i/360)];
    r = [r v];
end
	\end{lstlisting}
	Can you think of how you might get a list of vectors on the unit circle whose $x$ coordinate is less than $0.6643$?





\end{document}
