\documentclass[letter]{article}
\usepackage{amsmath}
\usepackage{amsfonts}
\usepackage{amssymb}
\usepackage{ifthen}
\usepackage{fancyhdr}
\usepackage{enumitem}

%%%
% Set up the margins to use a fairly large area of the page
%%%
\oddsidemargin=.2in
\evensidemargin=.2in
\textwidth=6in
\topmargin=0in
\textheight=9.0in
\parskip=.07in
\parindent=0in
\pagestyle{fancy}

%%%
% Set up the header
%%%
\newcommand{\setheader}[6]{
	\lhead{{\sc #1}\\{\sc #2} ({\small \it \today})}
	\rhead{
		{\bf #3} 
		\ifthenelse{\equal{#4}{}}{}{(#4)}\\
		{\bf #5} 
		\ifthenelse{\equal{#6}{}}{}{(#6)}%
	}
}

%%%
% Set up some shortcut commands
%%%
\newcommand{\R}{\mathbb{R}}
\newcommand{\N}{\mathbb{N}}
\newcommand{\Z}{\mathbb{Z}}
\newcommand{\Proj}{\mathrm{proj}}
\newcommand{\Perp}{\mathrm{perp}}
\newcommand{\proj}{\mathrm{proj}}
\newcommand{\Span}{\mathrm{span}}
\newcommand{\Null}{\mathrm{null}}
\newcommand{\Rank}{\mathrm{rank}}
\newcommand{\mat}[1]{\begin{bmatrix}#1\end{bmatrix}}
\renewcommand{\d}{\mathrm{d}}

%%%
% This is where the body of the document goes
%%%
\begin{document}
	\setheader{Math 281-1}{Homework 6}{Thursday, November 3}{}{}{}
	\begin{enumerate}
		\item Let $g:\R^2\to\R$ be some unknown function and let $\mathcal S\subseteq \R^3$ be the surface
			described by $z=g(x,y)$.  You know the following information:  the point
			$\vec A=(2,-3,1)\in \mathcal S$ and the tangent plane to $\mathcal S$ at the point
			$\vec A$ has normal vector $\vec n=(7,1,11)$.
			\begin{enumerate}
				\item Give an equation for the tangent plane to $\mathcal S$ at $\vec A$.  Is your
					equation the only equation describing the tangent plane?
				\item Write down a linear approximation to $g$ at the point $(2,-3)$.  Is
					your linear approximation the only linear approximation to $g$ at the point $(2,-3)$?
				\item Compute $\nabla g(2,-3)$.
			\end{enumerate}
		\item Let $f(x,y)=(x-3)^3-7y^2$, let $\vec a=(a_x,a_y)=(1,2)$, and let $\mathcal S\subseteq \R^3$ be the surface given
			by the equation $z=f(x,y)$.
		\begin{enumerate}
			\item Write $\mathcal S$ in set-builder notation.
			\item Draw level curves for $\mathcal S$.
			\item Let $\vec A = (a_x,a_y, f(a_x,a_y))$.  Compute four tangent vectors
				to $\mathcal S$ at the point $\vec A$ such that none of your four
				tangent vectors is parallel to another.  Call your vectors $\vec d_1$,
				$\vec d_2$, $\vec d_3$, $\vec d_4$.
			\item Geometrically, what is the object
				\[
					\mat{x\\y \\z}=t\vec d_1+s\vec d_2+r\vec d_3+q\vec d_4+\vec A
				\]
				where $t,s,r,q$ range over $\R$?
			\item Let $T_{\vec A}$ be the tangent plane to the surface $\mathcal S$ at the point $\vec A$.
				Write down $T_{\vec A}$ in normal form using a normal vector of $\vec n_1=\vec d_1\times \vec d_2$
				and using a normal vector of $\vec n_2=\vec d_3\times \vec d_4$.  Are your equations the same?
				Do they describe the same object?
			\item Solve both equations you computed above for $z$.  Now are the equations the same?  Can you
				rewrite these equations using $\nabla f$?
		\end{enumerate}
		\item \begin{enumerate}
			\item Let $\mathcal P\subseteq \R^3$ be a plane. Prove that there is at most one triplet
			of numbers $(a,b,c)$ so that $z=ax+by+c$ describes the plane $\mathcal P$.  
			(\emph{Hint: when trying to show something is unique, it is often useful to 
			assume there are two such things and then show that actually those two things were the
			same all along!})
			\item Consider the statement: \emph{For any plane $\mathcal P\subseteq \R^3$, you can always find numbers
				$a,b,c$ so that $\mathcal P$ is uniquely described by the equation $z=ax+by+c$.}
				Is this statement true or false?  Explain.
		\end{enumerate}
		\item {\sc Multivariable chain rule}.
			The idea of this problem is to discover and prove the multi-variable chain rule.
			For this problem, you may assume that the linear approximation to a differentiable 
			function $f:\R^n\to\R$
			at the point $\vec a$ is given by $L_{\vec a}(\vec x)=f(\vec a)+\nabla f(\vec a)\cdot (\vec x-\vec a)$.  However,
			please don't use any formula for the multi-variable chain rule that is in your notes. If
			we use that formula, the derivation of the chain rule would be circular!
			\begin{enumerate}
				\item Let $L(x,y)=ax+by+c$ for some unknown constants $a,b,c\in\R$.  Notice
					the graph of $z=L(x,y)$ is a plane.  Let $\vec p(t) = (p_x(t), p_y(t))$
					be a parameterization of a path in $\R^2$.

					Compute $(L\circ \vec p)'(t_0)$.  Express your answer as dot product
					involving $\nabla L$ and $\vec p\,'$.
				\item Let $g:\R^2\to\R$ be an unknown but differentiable function and let $\vec a\in\R^2$.
					Write an equation for the linear approximation, $L_{\vec a}$, of $g$ at the point $\vec a$.
				\item The great thing about linear approximations to differentiable functions 
					is that not only do they capture all
					information about the value of a function at a point, they capture all information
					about the first derivative of a function at a point.  That is, if
					$\vec p:\R\to\R^2$ is a parameterization of a path so that $\vec p(t_0)=\vec a$
					and $L_{\vec a}$ is a linear approximation of $g$ at the point $\vec a$, then
					\[
						(g\circ \vec p)'(t_0) = (L_{\vec a}\circ \vec p)'(t_0).
					\]
					Use this equality and your results from part (a) and (b) to write down a formula
					for $(g\circ \vec p)'(t_0)$ involving $\nabla g$ and $\vec p\,'$.
				\item Complete the formula
					\[
						(g\circ \vec p)'(t) = ?
					\]
					where $?$ consists only of the symbols $\nabla g$, $\vec p$, $\vec p\,'$, $t$ and whatever
					math operations you want (multiplication, dot product, composition, etc.).  In particular,
					$\vec a$ should \emph{not} show up in your formula.  Congratulations,
					you've discovered the multivariable chain rule!
				\item The professor across the hall from me claims the multivariable chain rule is
					\[
						(g\circ \vec p)'(t) = \Big[D_{\vec p(t)}g\Big](\vec p\,'(t)).
					\]
					Is she right? Explain.
			\end{enumerate}
%		\item {\sc Tangents all the time}.
%			\begin{enumerate}
%				\item Come up with equations three parameterizations
%					$\vec a:\R\to\R^2$, $\vec b:\R\to\R^2$, and $\vec c:\R\to\R^2$
%					that all parameterise \emph{different} paths in $\R^2$ but that satisfy
%					$\vec a(0)=\vec b(0)=\vec c(0)=(0,0)$ and $\vec a\,\!'(0)=\vec b\,'(0)=\vec c\,'(0)=(1,0)$.
%				\item Let $f(x,y)=(x-3)^3-7y^2$ and let $\mathcal S\subseteq \R^3$ be the surface described by $z=f(x,y)$.
%					Use $\vec a,\vec b,\vec c$ from part (a) to come up with three parameterizations $\vec A,\vec B,\vec C$
%					of paths contained in $\mathcal S$ that pass through the point $(0,0,f(0,0))$.  (You may use
%					the notation $\vec a_x(t)$ to refer to the $x$ component of $\vec a(t)$, etc., to make
%					your life easier).
%				\item Using a computer, create two plots.
%					One one, plot $\mathcal S$. On the other, plot the paths parameterized by $\vec A,\vec B$, and $\vec C$
%					(for some reasonable range of $t$ values).  (You can use the {\sc Matlab}/{\sc Octave} command
%					{\tt plot3} to make 3d plots of curves.  {\tt plot3(xs,ys,zs, xs2,ys2,zs2)} will plot two
%					curves in the same graph.)
%				\item Compute $\vec A'(0), \vec B'(0),$ and $\vec C'(0)$.  How do they compare?
%			\end{enumerate}
	\end{enumerate}

\end{document}
