\documentclass[letter]{article}
\usepackage{amsmath}
\usepackage{amsfonts}
\usepackage{amssymb}
\usepackage{amsthm}
\usepackage{mdframed}
\usepackage{ifthen}
\usepackage{fancyhdr}
\usepackage[usenames,dvipsnames,svgnames,table]{xcolor}
\usepackage{tikz}
\usepackage{changepage}
\graphicspath{{images/}}
\usepackage[hidelinks]{hyperref}

%%%
% Set up the margins to use a fairly large area of the page
%%%
\oddsidemargin=.2in
\evensidemargin=.2in
\textwidth=6in
\topmargin=0in
\parskip=.07in
\parindent=0in
\pagestyle{fancy}

\expandafter\def\expandafter\quote\expandafter{\quote\sf\color{DarkGreen}}

%%%
% Set up the header
%%%
\newcommand{\setheader}[2]{
	\lhead{{\sc #1}\\{\sc #2} %({\small \it \today})
	}
}

%%%
% Set up some shortcut commands
%%%
\newcommand{\R}{\mathbb{R}}
\newcommand{\N}{\mathbb{N}}
\newcommand{\Z}{\mathbb{Z}}
\newcommand{\Q}{\mathbb{Q}}
\newcommand{\Proj}{\mathrm{proj}}
\newcommand{\Perp}{\mathrm{perp}}
\newcommand{\Span}{\mathrm{span}}
\newcommand{\Null}{\mathrm{null}}
\newcommand{\Rank}{\mathrm{rank}}
\newcommand{\Range}{\mathrm{range}}
\newcommand{\Det}{\mathrm{det}}
\newcommand{\mat}[1]{\begin{bmatrix}#1\end{bmatrix}}
\newcommand{\Rref}{\mathrm{rref}}
\renewcommand{\d}{\mathrm{d}}

\DeclareMathOperator{\arccot}{arccot}

\newenvironment{answer}{
	\begin{adjustwidth}{8mm}{} \vspace{2mm}}{\end{adjustwidth} \vspace{2mm}
}

\theoremstyle{plain}
\newtheorem*{theorem}{Theorem}
\newtheorem*{lemma}{Lemma}

\theoremstyle{definition}
\newtheorem*{definition}{Definition}

\theoremstyle{remark}
\newtheorem*{claim}{Claim}

%%%
% This is where the body of the document goes
%%%
\begin{document}
	\setheader{Math 281-2}{Stokes's Theorem and Divergence Theorem}
	
	Problems are from Stewart's \textit{Essential Calculus: Early Transcendentals 2nd Ed.} or from
	Evans's \textit{ISP Mathematics 281 Volume I}. 	
	
	\begin{enumerate}
		\item	Stewart 13.8 1, 3
		\item	Stewart 13.8 5, 6, 7
		\item	Stewart 13.8 9 
		\item	Stewart 13.8 11, 13
		\item	Stewart 13.8 15
		\item	Stewart 13.8 17
		\item	Evans 5.9 1
		\item	Evans 5.9 3
		\item	Evans 5.9 6
		\item	Evans 5.9 8
		\item	Stewart 13.9 1, 3
		\item	Stewart 13.9 5, 7, 9, 13
		\item	Stewart 13.9 17
		\item	Stewart 13.9 19
		\item	Stewart 13.9 24
		\item	Stewart 13.9 25
		\item	Evans 5.5 1
		\item	Evans 5.5 2
		\item	Evans 5.5 3
		\item	Evans 5.5 5
		\item	Evans 5.5 7
	\end{enumerate}
\end{document}
