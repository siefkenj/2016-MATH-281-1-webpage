\documentclass[letter]{article}
\usepackage{amsmath}
\usepackage{amsfonts}
\usepackage{amssymb}
\usepackage{amsthm}
\usepackage{mdframed}
\usepackage{ifthen}
\usepackage{fancyhdr}
\usepackage[usenames,dvipsnames,svgnames,table]{xcolor}
\usepackage{tikz}
\usepackage{changepage}
\graphicspath{{images/}}
\usepackage[hidelinks]{hyperref}

%%%
% Set up the margins to use a fairly large area of the page
%%%
\oddsidemargin=.2in
\evensidemargin=.2in
\textwidth=6in
\topmargin=0in
\parskip=.07in
\parindent=0in
\pagestyle{fancy}

\expandafter\def\expandafter\quote\expandafter{\quote\sf\color{DarkGreen}}

%%%
% Set up the header
%%%
\newcommand{\setheader}[2]{
	\lhead{{\sc #1}\\{\sc #2} %({\small \it \today})
	}
}

%%%
% Set up some shortcut commands
%%%
\newcommand{\R}{\mathbb{R}}
\newcommand{\N}{\mathbb{N}}
\newcommand{\Z}{\mathbb{Z}}
\newcommand{\Q}{\mathbb{Q}}
\newcommand{\Proj}{\mathrm{proj}}
\newcommand{\Perp}{\mathrm{perp}}
\newcommand{\Span}{\mathrm{span}}
\newcommand{\Null}{\mathrm{null}}
\newcommand{\Rank}{\mathrm{rank}}
\newcommand{\Range}{\mathrm{range}}
\newcommand{\Det}{\mathrm{det}}
\newcommand{\mat}[1]{\begin{bmatrix}#1\end{bmatrix}}
\newcommand{\Rref}{\mathrm{rref}}
\renewcommand{\d}{\mathrm{d}}

\DeclareMathOperator{\arccot}{arccot}

\newenvironment{answer}{
	\begin{adjustwidth}{8mm}{} \vspace{2mm}}{\end{adjustwidth} \vspace{2mm}
}

\theoremstyle{plain}
\newtheorem*{theorem}{Theorem}
\newtheorem*{lemma}{Lemma}

\theoremstyle{definition}
\newtheorem*{definition}{Definition}

\theoremstyle{remark}
\newtheorem*{claim}{Claim}

%%%
% This is where the body of the document goes
%%%
\begin{document}
	\setheader{Math 281-1}{Polar, Cylindrical, and Spherical Coordinates}
	
	Problems are from Stewart's \textit{Essential Calculus: Early Transcendentals 2nd Ed.} or from Evans's \textit{ISP Mathematics 281 Volume I}. 	
	
	\begin{enumerate}
		\item Stewart 12.3 1-4
		\item Stewart 12.3 7, 9
		\item Stewart 12.3 13, 15
		\item Stewart 12.3 21
		\item Stewart 12.3 23, 25
		\item Stewart 12.3 29
		\item Stewart 12.6 15
		\item Stewart 12.6 17, 19, 23
		\item Stewart 12.7 17
		\item Stewart 12.7 21, 23, 28
		\item Stewart 12.7 33
		\item Stewart 12.7 37, 39
		\item Evans 4.3 2
		\item Evans 4.3 3
		\item Evans 4.3 4
		\item Evans 4.5 5
		\item Evans 4.5 7
		\item Evans 4.6 3
		\item Evans 4.5 7 (No Mass or Center of Mass)
		\item Evans 4.5 10
	\end{enumerate}
\end{document}