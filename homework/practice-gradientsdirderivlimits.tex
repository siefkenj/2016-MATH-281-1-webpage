\documentclass[letter]{article}
\usepackage{amsmath}
\usepackage{amsfonts}
\usepackage{amssymb}
\usepackage{amsthm}
\usepackage{mdframed}
\usepackage{ifthen}
\usepackage{fancyhdr}
\usepackage[usenames,dvipsnames,svgnames,table]{xcolor}
\usepackage{tikz}
\usepackage{changepage}
\graphicspath{{images/}}
\usepackage[hidelinks]{hyperref}

%%%
% Set up the margins to use a fairly large area of the page
%%%
\oddsidemargin=.2in
\evensidemargin=.2in
\textwidth=6in
\topmargin=0in
\parskip=.07in
\parindent=0in
\pagestyle{fancy}

\expandafter\def\expandafter\quote\expandafter{\quote\sf\color{DarkGreen}}

%%%
% Set up the header
%%%
\newcommand{\setheader}[2]{
	\lhead{{\sc #1}\\{\sc #2} %({\small \it \today})
	}
}

%%%
% Set up some shortcut commands
%%%
\newcommand{\R}{\mathbb{R}}
\newcommand{\N}{\mathbb{N}}
\newcommand{\Z}{\mathbb{Z}}
\newcommand{\Q}{\mathbb{Q}}
\newcommand{\Proj}{\mathrm{proj}}
\newcommand{\Perp}{\mathrm{perp}}
\newcommand{\Span}{\mathrm{span}}
\newcommand{\Null}{\mathrm{null}}
\newcommand{\Rank}{\mathrm{rank}}
\newcommand{\Range}{\mathrm{range}}
\newcommand{\Det}{\mathrm{det}}
\newcommand{\mat}[1]{\begin{bmatrix}#1\end{bmatrix}}
\newcommand{\Rref}{\mathrm{rref}}
\renewcommand{\d}{\mathrm{d}}

\DeclareMathOperator{\arccot}{arccot}

\newenvironment{answer}{
	\begin{adjustwidth}{8mm}{} \vspace{2mm}}{\end{adjustwidth} \vspace{2mm}
}

\theoremstyle{plain}
\newtheorem*{theorem}{Theorem}
\newtheorem*{lemma}{Lemma}

\theoremstyle{definition}
\newtheorem*{definition}{Definition}

\theoremstyle{remark}
\newtheorem*{claim}{Claim}

%%%
% This is where the body of the document goes
%%%
\begin{document}
	\setheader{Math 281-1}{Gradients, Directional Derivatives, and Limits}
	
	Problems are from Stewart's \textit{Essential Calculus: Early Transcendentals 2nd Ed.} or from Evans's \textit{ISP Mathematics 281 Volume I}. 	
	
	\textbf{Limits}
	\begin{enumerate}
		\item Stewart 11.2 3, 5, 11, 15
		\item Stewart 11.2 17
		\item Stewart 11.2 29, 31
		\item Evans 3.2 1, 2
	\end{enumerate}	
	\textbf{Directional Derivatives and Gradients}
	\begin{enumerate}
		\item Stewart 11.6 1, 2
		\item Stewart 11.6 3-6
		\item Stewart 11.6 15, 17
		\item Stewart 11.6 19
		\item Stewart 11.6 21
		\item Stewart 11.6 25
		\item Stewart 11.6 29
		\item Stewart 11.6 30
		\item Stewart 11.6 31, 33, 35
		\item Stewart 11.6 42
		\item Stewart 11.6 45
		\item Evans 3.3 1
		\item Evans 3.3 2
		\item Evans 3.3 4
		\item Evans 3.4 4
		\item Evans 3.4 6
		\item Evans 3.4 8 
	\end{enumerate}
\end{document}