\documentclass[letter]{article}
\usepackage{amsmath}
\usepackage{amsfonts}
\usepackage{amssymb}
\usepackage{ifthen}
\usepackage{fancyhdr}
\usepackage{enumitem}

%%%
% Set up the margins to use a fairly large area of the page
%%%
\oddsidemargin=.2in
\evensidemargin=.2in
\textwidth=6in
\topmargin=0in
\textheight=9.0in
\parskip=.07in
\parindent=0in
\pagestyle{fancy}

%%%
% Set up the header
%%%
\newcommand{\setheader}[6]{
	\lhead{{\sc #1}\\{\sc #2} ({\small \it \today})}
	\rhead{
		{\bf #3} 
		\ifthenelse{\equal{#4}{}}{}{(#4)}\\
		{\bf #5} 
		\ifthenelse{\equal{#6}{}}{}{(#6)}%
	}
}

%%%
% Set up some shortcut commands
%%%
\newcommand{\R}{\mathbb{R}}
\newcommand{\N}{\mathbb{N}}
\newcommand{\Z}{\mathbb{Z}}
\newcommand{\Proj}{\mathrm{proj}}
\newcommand{\Perp}{\mathrm{perp}}
\newcommand{\proj}{\mathrm{proj}}
\newcommand{\Span}{\mathrm{span}}
\newcommand{\Null}{\mathrm{null}}
\newcommand{\Rank}{\mathrm{rank}}
\newcommand{\mat}[1]{\begin{bmatrix}#1\end{bmatrix}}

%%%
% This is where the body of the document goes
%%%
\begin{document}
	\setheader{Math 281-1}{Homework 3}{Due: Friday, October 14}{}{}{}
	\begin{enumerate}
		\item Let $L\subset \R^3$ be the line that passes through the points $(0,0,0)$ and $(1,2,3)$
			and let $E\subset \R^2$ be ellipse, centered at the origin, with major axis along
			the $x$-axis, minor axis along the $y$-axis, major radius of $2$ and a minor radius of $1$.
			\begin{enumerate}
				\item Find three different parameterizations of $L$, $\vec p_1,\vec p_2,\vec p_3$ where
					$\vec p_1$ is an arc-length parameterization, $\vec p_2$ is a parameterization
					that moves at speed $2$, and $\vec p_3$ is a parameterization that passes through
					the point $(0,0,0)$ at speed 1 and $(1,2,3)$ at speed 2.  Make sure to specify the domains
					of each parameterization.
				\item Find a parameterization of $L$ by $(0,1)$ (the open unit interval).  Can you find a
					parameterization of $L$ by $[0,1]$ (the closed unit interval)?  Explain.
				\item Find a parameterization of $E$.  Make sure to specify its domain.
				\item Find a parameterization $\vec p$ of $E$ by $[0,1)$ with the added property
						that 
						\[
							\|\vec p'(0)\|=\lim_{t\to 1^-}\|\vec p'(t)\|=1.
						\] In your answer,
						you may use $C$ to stand for the circumference of the ellipse $E$.
					(\emph{Hint: there are many answers; some of them involve piecewise functions}.)
			\end{enumerate}
		\item An \emph{isometric parameterization} of a 2D surface $S$ is a parameterization $p(t,s)$
			that is length preserving \emph{and} area preserving.
			That is, the speed with respect to the first variable
			is $1$, the speed with respect to the second variable is $1$, and the area of the image
			of the square with corners $(\alpha,\beta),(\alpha+1,\beta),(\alpha,\beta+1),(\alpha+1,\beta+1)$ is 1.

			Consider the surface $S\subset \R^3$ parameterized by \[
				s(x,y) = (x^2,y, |y|^{3/2}).
			\]
			Produce an isometric parameterization $p:\R^2\to\R^3$ and verify each 
			property of the parameterization. (Hint, use a computer to visualize the surface
			and imagine it as a piece of paper.  Now choose easy direction vectors to 
			parameterize with.  Also, don't forget that $\|\vec a\times \vec b\|$ gives you
			the area of the parallelogram with sides $\vec a$ and $\vec b$.)

		\item In homework 2, you developed a theory of line integrals working directly with limits
			and Riemann sums.  Most textbooks present the theory of line integrals as follows:
			\begin{quote}
				Suppose $C\subset \R^2$ is a one dimensional curve with parameterization
				$\vec p:[a,b]\to \R^2$, and let $f:\R^2\to \R$ be a scalar-valued function.
				Then,
				\[
					\int_C f = \int_a^b f(\vec p(t))\|\vec p'(t)\|\mathrm{d}t,
				\]
				where $\int_C f$ is the signed area above the curve $C$ and below the surface
				$(x,y,f(x,y))$.
			\end{quote}

			Your task is to explain where this formula comes from using the foundational ideas of integral
			calculus: Riemann sums and limits.  You may work with a partner and turn in one writeup
			between the two of you, or you may work individually. 
			Your explanation should be typed in \LaTeX{} and include
			the following:
			\begin{enumerate}
				\item justification for any Riemann sums or limits you use;
				\item at least one example of a line integral;
				\item a discussion (it can be brief, or extended) about the special case of arclength
					parameterizations in relation to line integrals;
				\item appropriate citations if you used external resources (the \verb|\footnote{...}|
					command is your friend!).
			\end{enumerate}
			You are strongly encouraged to include pictures.  Your target audience is a student
			who just completed integral calculus.  
			Your writeup will be graded both on correctness
			and on how well you explain/present the topic.
	\end{enumerate}

\end{document}
