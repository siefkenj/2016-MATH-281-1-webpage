\documentclass[letter]{article}
\usepackage{amsmath}
\usepackage{amsfonts}
\usepackage{amssymb}
\usepackage{ifthen}
\usepackage{fancyhdr}
\usepackage{enumitem}

%%%
% Set up the margins to use a fairly large area of the page
%%%
\oddsidemargin=.2in
\evensidemargin=.2in
\textwidth=6in
\topmargin=0in
\textheight=9.0in
\parskip=.07in
\parindent=0in
\pagestyle{fancy}

%%%
% Set up the header
%%%
\newcommand{\setheader}[6]{
	\lhead{{\sc #1}\\{\sc #2} ({\small \it \today})}
	\rhead{
		{\bf #3} 
		\ifthenelse{\equal{#4}{}}{}{(#4)}\\
		{\bf #5} 
		\ifthenelse{\equal{#6}{}}{}{(#6)}%
	}
}

%%%
% Set up some shortcut commands
%%%
\newcommand{\R}{\mathbb{R}}
\newcommand{\N}{\mathbb{N}}
\newcommand{\Z}{\mathbb{Z}}
\newcommand{\Proj}{\mathrm{proj}}
\newcommand{\Perp}{\mathrm{perp}}
\newcommand{\proj}{\mathrm{proj}}
\newcommand{\Span}{\mathrm{span}}
\newcommand{\Null}{\mathrm{null}}
\newcommand{\Rank}{\mathrm{rank}}
\newcommand{\mat}[1]{\begin{bmatrix}#1\end{bmatrix}}
\renewcommand{\d}{\mathrm{d}}

%%%
% This is where the body of the document goes
%%%
\begin{document}
	\setheader{Math 281-1}{Homework 5}{Due: Thursday, October 27}{}{}{}

	\begin{enumerate}
		\item You are designing an oval-shaped race track.  The race track consists of two
			straight stretches of length 500 meters and two semicircles of radius 100 meters.
			On a straight stretch, a car attains a maximum speed of 100 meters per second (near the end of the straight stretch).
			When traveling along a curved stretch, a car experiences a constant deceleration
			and leaves the curved stretch going 50 meters per second.

			Since you don't want cars to slide off the track as they travel along the curved part
			of the track, you plan on angling the curved part of the road (like a real race track).
			Assuming there is no friction (so the only forces a car is subjected to are the acceleration
			due to its movement and the acceleration due to gravity), compute the angle of elevation of
			the curved part of the race track so that a car won't slide off (Hint, this won't be constant!).

			You may assume that the acceleration due to gravity is $g$ meters per second squared.

		\item Let $f(x,y)=-x^2+y$.
			\begin{enumerate}
				\item Compute the level curves $f(x,y)=0$, $f(x,y)=1$, and $f(x,y)=2$.  
					Describe each level curve as a subset of $\R^2$ using set-builder notation.
				\item Sketch $f$ using level curves.  Describe in words what the graph of 
					$z=f(x,y)$ looks like.
				\item A \emph{set-valued function} is a function that outputs sets instead
					of numbers or vectors.  For example, $g(t) = \{x:x\in [0,t)\}$ is
					a set-valued function outputting a different set for every
					$t$.  Write down a set-valued function $L:\R\to\{\text{curves}\}$
					so that $L(h)$ outputs the level curve $f(x,y)=h$.  
			\end{enumerate}

		\item 
			\begin{enumerate}
				\item The level curves of the function $f:\R^2\to [0,\infty)$ are given 
					in polar coordinates by the function
					$L(h)=\{(\theta, h):\theta\in [0,2\pi)\}$.  That is, 
						$L(h)$ gives the level curve $f(x,y)=h$.  Graph the level curves
						of $f$ and then make a 3d
						sketch of the surface $(x,y,f(x,y))$.
					\item The level curves of $g:\R^2\to [1,\infty)$ are given in polar coordinates by the function
							$L(h)=\{(\theta, \sin(\theta)/h):\theta\in(0,\pi)\}$.
							Graph the level curves of $g$ and then make a 3d sketch of $(x,y,g(x,y))$.
			\end{enumerate}
		\item Determine the following limits if they exist, otherwise show they don't exist.
			\begin{enumerate}
				\item $\displaystyle\lim_{(x,y)\to(0,0)} x+y$
				\item $\displaystyle\lim_{(x,y)\to(0,0)} \frac{x^2-y^2}{x^2+y^2}$
				\item $\displaystyle\lim_{(x,y)\to(0,0)} \frac{x^2y}{x^4+y^2}$
			\end{enumerate}

		\item We've seen that for a function $f$ and a direction $\vec u$, $\nabla f(\vec a)\cdot \vec u$ gives
			the directional derivative at $\vec a$ in the direction $\vec u$ \emph{if $f$ is differentiable}.
			This last part is key.

			Let $f(x,y) = \frac{x^2y}{x^2+y^2}$ if $(x,y)\neq (0,0)$ and $f(0,0) = 0$.  Use the definition
			of the directional derivative to compute the directional derivative of $f$ at $(0,0)$ in the direction
			$\vec u$.  Can the directional derivative be written as $\nabla f(0,0)\cdot \vec u$ for all $\vec u$?
			How about for some $\vec u$?

			Show that $f$ is not differentiable at $(0,0)$.
	\end{enumerate}
\end{document}
